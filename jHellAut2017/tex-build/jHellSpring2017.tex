% ****** Start of file apssamp.tex ******
%
%   This file is part of the APS files in the REVTeX 4.1 distribution.
%   Version 4.1r of REVTeX, August 2010
%
%   Copyright (c) 2009, 2010 The American Physical Society.
%
%   See the REVTeX 4 README file for restrictions and more information.
%
% TeX'ing this file requires that you have AMS-LaTeX 2.0 installed
% as well as the rest of the prerequisites for REVTeX 4.1
%
% See the REVTeX 4 README file
% It also requires running BibTeX. The commands are as follows:
%
%  1)  latex apssamp.tex
%  2)  bibtex apssamp
%  3)  latex apssamp.tex
%  4)  latex apssamp.tex
%
\documentclass[
reprint,
%superscriptaddress,
%groupedaddress,
%unsortedaddress,
%runinaddress,
%frontmatterverbose, 
%preprint,
%showpacs,preprintnumbers,
%nofootinbib,
%nobibnotes,
%bibnotes,
 amsmath,amssymb,
 aps,
 prl,
%pra,
%prb,
%rmp,
%prstab,
%prstper,
%floatfix,
]{revtex4-1}

\newcommand{\partDeriv}[2]{\frac{\partial #1}{\partial #2}}
\DeclareMathOperator\erf{erf}

\usepackage{graphicx}% Include figure files
\usepackage{dcolumn}% Align table columns on decimal point
\usepackage{bm}% bold math
\usepackage{hyperref}
%\usepackage{hyperref}% add hypertext capabilities
%\usepackage[mathlines]{lineno}% Enable numbering of text and display math
%\linenumbers\relax % Commence numbering lines

%\usepackage[showframe,%Uncomment any one of the following lines to test 
%%scale=0.7, marginratio={1:1, 2:3}, ignoreall,% default settings
%%text={7in,10in},centering,
%%margin=1.5in,
%%total={6.5in,8.75in}, top=1.2in, left=0.9in, includefoot,
%%height=10in,a5paper,hmargin={3cm,0.8in},
%]{geometry}

\begin{document}

%\preprint{APS/123-QED}

\title{On The Diffusion of Sticky Particles in 1-D}% Force line breaks with \\
%\thanks{A footnote to the article title}%

\author{Joshua DM Hellier}
 \email{J.D.M.Hellier@sms.ed.ac.uk}
% \altaffiliation[Also at ]{Physics Department, XYZ University.}%Lines break automatically or can be forced with \\
\author{Graeme J Ackland}%
 \email{G.J.Ackland@ed.ac.uk}
\affiliation{%
 SUPA, School of Physics and Astronomy, University of Edinburgh, Mayfield Road, Edinburgh EH9 3JZ, United Kingdom
}%

%\collaboration{\noaffiliation}

%
%\author{Charlie Author}
% \homepage{http://www.Second.institution.edu/~Charlie.Author}
%\affiliation{
% Second institution and/or address\\
% This line break forced% with \\
%}%
%\affiliation{
% Third institution, the second for Charlie Author
%}%
%\author{Delta Author}
%\affiliation{%
% Authors' institution and/or address\\
% This line break forced with \textbackslash\textbackslash
%}%

%\collaboration{CLEO Collaboration}%\noaffiliation

\date{\today}% It is always \today, today,
             %  but any date may be explicitly specified

\begin{abstract}
The 1D Ising model is the simplest Hamiltonian-based model in statistical
mechanics. The simplest interacting particle process, is the Symmetric
Exclusion Process (SEP), a  1D lattice gas of particles that hop
symmetrically and cannot overlap.  Combining the two gives a model for
sticky particle diffusion, SPM, which is described here.  SPM dynamics
are based on SEP with short-range interaction, allowing flow due to
non-equilibrium boundaries conditions.  We prove that SPM is also a
detailed-balance respecting, particle-conserving,  Monte Carlo description of the Ising
model.  Neither the Ising model nor SEP have a phase transition in 1D, but the SPM exhibits a non-equilibrium transition from a
diffusing to a blocked state as stickiness increases.  We present a
fully non-linear, analytic, mean-field solution, which has a crossover
from positive to negative diffusion constant.  Simulations in the
positive-diffusion region agree with the analytics. The negative
diffusion constant in fact indicates a breakdown of the mean-field
approximation, with close to zero flow and breaking into a two-phase
mixture, thus the mean field theory successfully predicts its own
demise.  The simplicity of the model suggests a wide range of
possible applications.

\iffalse
\begin{description}
\item[Usage]
Secondary publications and information retrieval purposes.
\item[PACS numbers]
May be entered using the \verb+\pacs{#1}+ command.
\item[Structure]
You may use the \texttt{description} environment to structure your abstract;
use the optional argument of the \verb+\item+ command to give the category of each item. 
\end{description}
\fi
\end{abstract}

%\pacs{Valid PACS appear here}% PACS, the Physics and Astronomy
                             % Classification Scheme.
%\keywords{Suggested keywords}%Use showkeys class option if keyword
                              %display desired
\maketitle


\iffalse
There are a great many natural phenomena which involve the diffusion of small particles through solids; interface problems, such as the growth of a titanium dioxide layer on the surface of titanium metal exposed to air, are good examples~\cite{tegner2015high}.
If we wish to answer questions such as why these interfaces grow, or how quickly, we really need to understand how particles diffuse through crystal lattices, especially in the case when they interact with each other.
In this paper we will introduce a very simple locally interacting exclusion model of this kind of diffusion, and we will explore the continuum-level implications of such a model.

We would intuitively expect that our titanium interface growth problem would involve the diffusion of oxygen atoms through titanium metal crystals. Once the concentration of oxygen is high enough, the medium becomes titanium dioxide.
The oxygen atoms do this primarily by hopping between the interstitial sites between the titanium atoms.
% Need citation
It is extremely energetically unfavourable for multiple oxygen atoms to occupy such a site,
%citation needed
therefore to a good approximation we may regard these oxygen atoms as excluding each other from these sites, just like in ASEP~\cite{sugden2007dynamically, liggett1985interacting}.
\fi

Lattice gases are a ubiquitous tool for modeling complex systems from biology to traffic~\cite{1742-5468-2011-07-P07007, Mobilia2007, tegner2015high, zhu2012atomic, DealGrove1965, MottCabrera1949, Buzzaccaro2007}. 
%One example is the oxidation process, as occurs in the growth of an oxide layer on a metallic surface. Almost all uncoated metallic objects
%[Al, Ti, Ca, Zr, Sn] are protected by such a thin, self-assembling layer, driven by the oxidation potential but arrested by the slow kinetics of transporting material through the oxide layer~\cite{tegner2015high, zhu2012atomic, DealGrove1965, MottCabrera1949}.
Analytically solvable cases involve non-interacting or excluding particles~\cite{ladd1988application, liggett1985interacting}, but in any real system of interest the moving objects interact. Many models tackle the situation where the diffusing
object interact with the substrate,
% find example
but despite the clear application-relevance there is surprisingly little work considering interactions between the moving particles themselves.  One reason for this is that the interactions introduce nonlinearities in analytical models, which makes them
challenging to solve, at least outside of limits in which they can be linearized. This is unfortunate because it is precisely these nonlinearities which introduce interesting behaviors such as discontinuities at the oxide-metal interface or
diffusion instability.

In this paper we will
investigate a simple one-dimensional model, the ``Sticky Particle Model'' or SPM, specified in Fig.~\ref{fig:rates}, which contains such an interaction, and we will explore the impact this has on particle behavior, in particular
when observed in the large-scale limit.
One might contrast
this approach (making a simple microscopic model and trying to learn from it about large-scale interface growth) with approaches such as the KPZ equation~\cite{PhysRevLett.56.889, PhysRevA.38.4271, Sasamoto2010} (where one analyses the extreme
large-scale dynamics using universality classes).
\iffalse
Next, let us assume that the lattice that the oxygen atoms move through is fairly rigid (i.e. that the titanium atoms are quite tightly bound and don't move that much),
and that the interactions between the oxygen atoms are quite short-ranged (as any electrostatic forces should be rapidly screened by the metal, thus the main interaction should be via short-range electrostatics and electron sea distortion).
Finally, we should note that even though a problem like interface growth happens in three-dimensional space, the problem is rotationally and translationally invariant in a plane perpendicular to the direction of growth; therefore the
interesting aspects of the problem are one-dimensional. Indeed, in anisotropic solids it is often the case that diffusion occurs much more rapidly along parallel chains than in other directions.
Putting these assumptions together, we are motivated to investigate the model described by the rates detailed in Figure~\ref{fig:rates}. It is essentially the symmetric exclusion model, only now the presence of an adjacent particle
causes the hopping rate to change; there also exists an isomorphism between this model and the Misanthrope Process~\cite{evansWaclaw2014}, although I have yet to find a use for it. We will henceforth refer to the model as the ``sticky particle model'', or SPM.
\fi
\begin{figure}
\vspace{1em}
\caption{\label{fig:rates} White circles indicate particles, dark circles indicate empty sites (vacancies). Particles randomly move into adjacent vacancies with rate $1$ (having rescaled time for notational convenience), unless there is a
particle behind the position they're moving from, in which case they move with rate $\lambda$; the state of the site next to the position the particle is moving into is irrelevant.
Particles also move to the left, with rates such that the whole model is totally symmetric.}
    \includegraphics[width=\linewidth]{../tex-src/images/newRates}
    \vspace{-3em}
\end{figure}
The SPM is based upon the symmetric exclusion process~\cite{sugden2007dynamically, Kollmann2003,  Lin2005, Hegde2014, Krapivsky2014, Imamura2017};
it differs from the original in that adjacent particles separate with rate $\lambda$
instead of their normal hopping rate, $1$. It is in fact a version of the KLS model~\cite{Katz1984, Zia2010} in 1-dimension without an applied field, which is itself similar to the dynamics used to analyze the Ising model by
Kawasaki~\cite{PhysRev.145.224}. It seems that this symmetric model has not been researched much (at least in terms of its dynamics) because the model with the applied field is so interesting; however, it seems that the simple symmetric model
exhibits complex unexpected behavior when a concentration gradient is applied. The quantity $\lambda$ parametrizes the ``stickiness'' of the particles; when $\lambda>1$, there is a tendency for particles to repel,
whilst $\lambda < 1$ represents attraction.
It is worth noting that the rates specified in Fig.~\ref{fig:rates} obey detailed balance,  %cite supplimentary materials
with an energy proportional to the number of particle-particle adjacencies
in the system. It seems that space of highly-local exclusion models is so tightly constrained in one-dimension that there is no option but to comply with the detailed balance condition.
%Could stick this in appendix maybe?

%One can show that there exists an isomorphism between this model and the Misanthrope Process~\cite{evansWaclaw2014}. This doesn't help us to solve the model in the kinds of steady flow cases we are mainly interested in due to
%difficulties with the boundary conditions, but it does allow us to show that this model does \textbf{not} exhibit explosive condensation~\cite{waclaw2012explosive}.



The model described in Fig.~\ref{fig:rates} is very simple, but numerical simulation shows that it is capable of a wide range of behaviors, such as those shown in Fig.~\ref{fig:flowPatterns}. We will discuss
these numerical results in more detail later, but first let us analyze the model behavior using analytic means.
Because this model contains interactions, the types of methods for the full analytic solution of SEP don't help us; thus, the best we can currently do is a mean field theory approximation.
Let the spacing between lattice sites be $a$, let $\tau_0$ be the free-particle hopping timescale, and the time-averaged (or ensemble-averaged, assuming ergodicity) occupation probability of the $i^{\mathrm{th}}$ lattice site be $\rho_i$.
We introduce $\zeta = 1 - \lambda $ here for convenience.
One may show that, in the mean-field approximation regime,
% maybe derive in appendix
\begin{align}
\begin{split}
 \tau_0 \partDeriv{\rho_i}{t} = &\left( 1-\rho_i \right) \left[ \left(1-\zeta\rho_{i-2} \right) \rho_{i-1} + \left(1-\zeta\rho_{i+2} \right) \rho_{i+1} \right] \\
 &- \rho_i \left[ 2 \zeta \rho_{i-1} \rho_{i+1}  - (3-\zeta)\left(\rho_{i-1} + \rho_{i+1}\right) + 2 \right].
 \end{split}
 \end{align}
Switching to the continuum limit by taking $a\rightarrow 0$, and neglecting $\mathcal{O}(a^4)$ terms, we may re-express this as a conserved flow $J$ as follows:
\begin{align}
 \partDeriv{\rho}{t} &= - \partDeriv{J}{x}, \\
 J &= -  D(\rho) \partDeriv{\rho}{x}, \\
 D(\rho) &= \frac{a^2}{\tau_0} \left[1 - \zeta \rho\left(4-3\rho\right) \right]. 
\end{align}
Thus, the MFT says that the particles should diffuse with a diffusion coefficient $A(\rho)$ which depends upon the local density.

In order to understand the implications of the MFT, let us consider some limits. As $\zeta \rightarrow 0$ (i.e. as the model becomes a simple exclusion model), $D \rightarrow \frac{a^2}{\tau_0}$. Likewise, in the
dilute limit $\rho \rightarrow 0$, $D \rightarrow \frac{ a^2}{\tau_0}$, reflecting the fact that it becomes a dilute lattice gas and therefore the interactions between particles become irrelevant as they never meet.
Conversely, in the full limit $\rho \rightarrow 1$, $D \rightarrow \frac{\lambda a^2}{\tau_0}$; this is because we now have a dilute gas of vacancies, which hop with rate $\frac{\lambda}{\tau_0}$.
One may observe that the continuum limit MFT has a symmetry under $\rho \mapsto \frac{4}{3} - \rho$; thus, the dynamics should be symmetric under a density profile reflection around $\rho = \frac{2}{3}$. This is where $D$ always
attains its extremal value, $ \frac{a^2}{\tau_0}\left[1 - \frac{4}{3}\zeta\right]$, hence for $\zeta>3/4$ the diffusion coefficient becomes negative in regions with
$\frac{2}{3} - \frac{\sqrt{\zeta\left(4\zeta - 3\right)}}{3\zeta} < \rho < \frac{2}{3} + \frac{\sqrt{\zeta\left(4\zeta - 3\right)}}{3\zeta}$.
Finally, it is possible to show that solutions to the continuum MFT containing domains with a negative diffusion coefficient are linearly unstable; thus, if we try to have a flow containing $\rho$ for which $D(\rho)<0$,
the density of the medium should gravitate towards a density for which $D(\rho)\sim 0$. Instead of observing ``backwards diffusion'' we would see an extremely slow flow or no flow at all. The MFT implies that the transition
to this critically slowly-flowing regime happens suddenly, like a phase transition: this can be checked with our numerics.

It is possible to solve the continuum MFT in a steady state on a finite domain, say $x\in(0, L)$. The continuity equation implies that $J(x)=J_0$, and by integrating both sides of that equation with respect to $x$ we find that
\begin{equation}
 J_0 (x-x_0) = -\frac{a^2}{\tau_0} \rho \left[1+\zeta \rho\left(\rho-2\right)\right],
\end{equation}
a cubic equation which can be solved to give $\rho(x)$. If we impose Dirichlet boundary conditions on this system, say $\rho(0)=\rho_0$ and $\rho(L)=\rho_L$, we find that
\begin{equation}
 J = \frac{a^2}{L \tau_0} \left[ \rho_0 - \rho_L + \zeta \left( \rho_0\left[\rho_0^2-2\right] - \rho_L\left[\rho_L^2-2\right] \right) \right].
\end{equation}
We may consider applying small concentration gradients across a block by setting $\rho_0 = \rho_M + \frac{1}{2}\delta\rho$ and $\rho_L = \rho_M - \frac{1}{2}\delta\rho$. Doing so, we find that the effective diffusion coefficient of the block
$D_\mathrm{Eff}=\partDeriv{J}{\delta\rho}\big|_{\delta\rho=0}$ obeys
\begin{equation}
\label{eq:MFTflow}
 \partDeriv{J}{\delta\rho}\bigg|_{\delta\rho=0} = \frac{a^2}{L \tau_0} \left[ 1 - \zeta\rho_M(4-3\rho_M) \right].
\end{equation}

We have derived MFT predictions about the SPM, and have indications about when those predictions might become invalid, which we tested numerically.
We chose to calculate using the \texttt{KMCLib}\cite{leetmaa2014kmclib} package, which implements the Kinetic Monte Carlo algorithm
(essentially the same as the Gillespie algorithm\cite{Gillespie1977})
on lattice systems. The codes used are kept here~\cite{jHellGitRepo}.
%\texttt{KMCLib} has the advantage that it is python-wrapped \texttt{C++}, and thus quite easy to use whilst at the same time being quite computationally efficient; thus it was fairly easy for us to carry out large numbers
%of differently-parametrised serial \texttt{KMCLib} jobs on the \texttt{Eddie3} computing cluster here at Edinburgh. 
As we have MFT predictions about flow in a block, we can simulate that situation using KMC. In the bulk, the transition rates are simply those described in Fig.~\ref{fig:rates}. At the boundaries
there are 2 layers of lattice sites what switch between being full and empty with rates such that the time-averaged occupation can be specified to match the desired boundary conditions; there are then chances for particles to appear
and disappear with rates depending upon the occupation of these boundary layers. In the end, the intention is that these boundaries should reproduce the effect of having particle reservoirs attached to the edges of the domain,
which is something we check
in the output by inspecting the time-averaged occupations of sites near the boundary. We have used this setup to explore three scenarios, discussed in the following sections. In each of these we refer to a boundary condition configuration
by $(\rho_0, \rho_L)$, with $\rho_0$ and $\rho_L$ being the bottom and top boundary densities respectively.
Measuring overall particle flow rate with such a setup is fairly simple, as we know how many particles enter and leave in a given time.
We can also calculate the time-averaged total number of particles in the system; this is done by updating a histogram of particle numbers
as the simulation progresses. In each of our calculations, we make the initial configuration by randomly filling the system with particles and vacancies in such a way that the initial density should be $\frac{1}{2}(\rho_0 + \rho_L)$, and then
run the system for a sufficient number of equilibration steps to destroy any initial transients.

The MFT suggests that a  transition from a steady flow regime to a critically slow flow regime might occur as the stickiness varies.
We test this by holding the boundary densities constant
whilst changing $\lambda$, and measuring the particle density as well as the mean, variance and skewness of the flow rate. If such a transition does indeed occur, we should expect to see a divergence in one of these moments.
We have done this with four sets of boundary conditions as shown in Fig.~\ref{fig:lambdaScans}. 
\iffalse
\begin{figure*}[h!]
\vspace{1em}
\caption{\label{fig:lambdaScans} Descriptive statistics of flow rates and average overall densities observed when varying $\lambda$ with fixed boundary densities $(\rho_0, \rho_L)$; data series are labelled in the plot.
In the case of the mean flow we have an MFT prediction, indicated by the solid line.
In each case we used systems of length $64$ (length $32$ gives similar results),
running them for $400000$ Gillispie steps for equilibration followed by $10000$ measurement runs of $1000$ steps interspersed with relaxation runs of $16000$
steps. This way we could gather statistics about flow rates and densities in a well-equilibrated system. Specifically, we generate a pool of $10000$ samples of flow rate and density,
from which we can calculate estimates of the descriptive statistics of both quantities.}
\begin{center}
 \begin{tabular}{c|c}
    \includegraphics[width=0.5\linewidth]{../tex-src/images/lambdaScan/newFlowMean} & \includegraphics[width=0.5\linewidth]{../tex-src/images/lambdaScan/newFlowVar} \\
    \hline
    \includegraphics[width=0.5\linewidth]{../tex-src/images/lambdaScan/newFlowSkew} & \includegraphics[width=0.5\linewidth]{../tex-src/images/lambdaScan/newDens} \\
    \end{tabular}
\end{center}
    \vspace{-0em}
\end{figure*}
\fi
\begin{figure}[h!]
\vspace{1em}
\caption{\label{fig:lambdaScans} Mean flow rate observed when varying $\lambda$ with fixed boundary densities $(\rho_0, \rho_L)$; data series are labeled in the plot.
The MFT prediction is indicated by the solid line.
In each case we used systems of length $64$ (length $32$ gives similar results),
running them for $400000$ Gillispie steps for equilibration followed by $10000$ measurement runs of $1000$ steps interspersed with relaxation runs of $16000$
steps. This way we could gather statistics about flow rates and densities in a well-equilibrated system. Specifically, we generate a pool of $10000$ samples of flow rate and density
from which we can calculate estimates of the descriptive statistics of both quantities; flow moments and the density data are included in the supplementary materials.
\vspace{1em}}
\includegraphics[width=0.9\linewidth]{../tex-src/images/lambdaScan/newFlowMean}
    \vspace{-2em}
\end{figure}

Firstly, we should note that we only have an MFT prediction for the flow rate as a function of $\lambda$, since $\rho(x)$ stops being unique when $\lambda$ drops below $\frac{1}{4}$,
and so the MFT lacks predictive power. For low-stickiness, when $\lambda>\frac{1}{4}$, the MFT is in good agreement with the simulations.
However, one of the key predictions of the MFT - that a sharp transition to a no-flow regime occurs when $\lambda$ becomes small enough (at least for 3 of the 4 sets of
boundary conditions we investigated here) - is not realized in our simulations. Indeed what seems to be happening is that the sharp transition has been smoothed out, as we
do not see any peaks or jumps in the flow rate variance or skewness (which we would expect to see if there was a transition). We suspect that this discrepancy is due to nontrivial correlations emerging between the particles, which the MFT
does not take account of. Alternatively, it could be that the continuum assumption is failing due to the finite-sized system filling with particles and blocking.
\iffalse 
As for the observed average density, for larger $\lambda$ the density approaches the average of the boundary densities,
and for small $\lambda$ the density approaches $1$ (which makes sense as the particles are very strongly attracted to each other, and so the system has a tendency to fill up); the exception to this it the case with extreme full/empty boundary
conditions, although in this case one might argue that the particles are ``sucked out'' of the system so rapidly at the empty end that the system never really has a chance to fill up. It is also worth noticing that this extreme case is the only
one in which the flow rate skewness does anything interesting; it is mostly positive, especially at low-$\lambda$, implying that most of the time the system is fairly static, but occasionally short-lived strong flows occur which end up causing
most of the bulk flow.
\fi

Another situation we can investigate has boundaries $(\rho_B, \rho_T) = (\rho_M + \frac{1}{2} \delta\rho, \rho_M - \frac{1}{2} \delta\rho)$ for some given $\rho_M$, where $\delta\rho$ and $\lambda$ are varied. As before, we calculated flow rate
moments and average densities, and the results are displayed in Fig.~\ref{fig:constDens}. 
\iffalse
\begin{figure*}[h!]
\vspace{1em}
\caption{\label{fig:constDens} Flow rate mean, flow variance and average overall densities observed when varying the difference $\delta\rho$ between the boundary concentrations
$(\rho_B, \rho_T) = (\rho_M + \frac{1}{2} \delta\rho, \rho_M - \frac{1}{2} \delta\rho)$ and $\lambda$. I chose $\rho_M=\frac{1}{2}$, as this gives us the biggest range of $\delta\rho$ to investigate.
These calculations were performed with the same run parameters (system length etc)
as above. The top left panel is the MFT prediction
for the flow rate, whilst top right shows the observed mean flow rate. The measured flow skewness and kurtosis are not displayed here as both signals were small and noisy, and didn't show anything particularly significant.}
\begin{center}
 \begin{tabular}{c|c}
    \includegraphics[width=0.5\linewidth]{../tex-src/images/constDens/newMftPred} & \includegraphics[width=0.5\linewidth]{../tex-src/images/constDens/newFlow} \\
    \hline
    \includegraphics[width=0.5\linewidth]{../tex-src/images/constDens/newVar} & \includegraphics[width=0.5\linewidth]{../tex-src/images/constDens/newDens} \\
    \end{tabular}
\end{center}
    \vspace{-0em}
\end{figure*}
\fi
\begin{figure}[h!]
\vspace{1em}
\caption{\label{fig:constDens} Flow rate mean observed when varying the difference $\delta\rho$ between the boundary concentrations
$(\rho_B, \rho_T) = (\rho_M + \frac{1}{2} \delta\rho, \rho_M - \frac{1}{2} \delta\rho)$ and $\lambda$ (The top panel is the MFT prediction
for the flow rate, whilst bottom shows the observed mean flow rate).
We chose $\rho_M=\frac{1}{2}$, as this gives us the biggest range of $\delta\rho$ to investigate.
These calculations were performed with the same run parameters (system length etc)
as above.}
\begin{center}
 \begin{tabular}{c}
    \includegraphics[width=0.98\linewidth]{../tex-src/images/constDens/newMftPred} \\
    \includegraphics[width=0.98\linewidth]{../tex-src/images/constDens/newFlow}
    \end{tabular}
\end{center}
    \vspace{-2.5em}
\end{figure}
The MFT prediction for the mean flow is generally in good agreement with the numerics, except in the very sticky regime in which flow is very small.
The simulations show no evidence of negative diffusion; rather the flow becomes critically slow for very sticky particles.
The higher moments of the flow (e.g. variance) do not show peaks, indicating that hard transitions are not occurring.
Finally, the density is very close to the average of the boundary densities until λ drops below 1/4, at which point the stickiness causes the system to fill.

Continuing to specify the boundary densities to be $(\rho_0, \rho_L) = (\rho_M + \frac{1}{2} \delta\rho, \rho_M - \frac{1}{2} \delta\rho)$ for some given $\rho_M$, we can keep $\delta\rho$ relatively small, so that $J$ varies approximately
linearly with $\delta\rho$; thus if we calculate $J$ for a series of small $\delta \rho$, we can perform linear regression to find $D_\mathrm{Eff}=\partDeriv{J}{\delta\rho}\big|_{\delta\rho=0}$, the effective diffusion coefficient.
Computing this for different $(\rho_M, \lambda)$ combinations yields results that can be compared with Eq.~\ref{eq:MFTflow}.
\begin{figure}[h!]
\vspace{1em}
\caption{\label{fig:diffCoef}
Comparison of effective diffusion coefficient $D$ in the MFT (top) and in direct simulation (bottom) as a function of density and stickiness.
The white region is where the MFT gives negative diffusion. The simulations used 124 sites averaged over $\sim 10^9$ steps at each of $12 \times 24 \times 16 $ $(\lambda, \rho_M, \delta \rho)$ combinations.
Full details in the supplementary materials.
\iffalse
The top contour plot shows the MFT prediction of the effective diffusion coefficient $D=\partDeriv{J}{\delta\rho}\big|_{\delta\rho=0}$ as a function of local density $\rho_M$ and $\lambda$;
we are only plotting where $0 \le D \le 1.2$, other regions are shown in white, including the region in which $D<0$, which would cause instabilities and so prevent a flow from actually occurring.
Bottom is our numerical calculation of $D(\rho_M, \lambda)$,
with exactly the same plotting ranges.
In this setup we ran the simulation for
$1.6\times10^8$ equilibration steps, followed by $10$ sets of alternating measurement and relaxation runs, of lengths $8\times10^7$ and $1.6\times10^7$ steps respectively. These results are consistent with calculations performed on smaller
systems, so we should be safe from finite-size effects.
\fi
}
\iffalse
\begin{center}[h!]
 \begin{tabular}{c@{\hspace{1em}}c}
    \includegraphics[width=0.5\linewidth]{../tex-src/images/newAnalFlow} & \includegraphics[width=0.5\linewidth]{../tex-src/images/newDataFlow} \\
    \includegraphics[width=0.5\linewidth]{../tex-src/images/newFlowDens} & \includegraphics[width=0.5\linewidth]{../tex-src/images/newFlowErr} 
    \end{tabular}
\end{center}
    \vspace{-3em}
\end{figure*}
\fi
%REMEMBER TO ACTUALLY PUT IT THERE!!!
\begin{center}
 \begin{tabular}{c}
    \includegraphics[width=0.98\linewidth]{../tex-src/images/newAnalFlow} \\
    \includegraphics[width=0.98\linewidth]{../tex-src/images/newDataFlow}
    \end{tabular}
\end{center}
    \vspace{-2em}
\end{figure}

One can compare the MFT prediction and the actual numerical results together as we have done in Fig.~\ref{fig:diffCoef}. We see that MFT and simulation agree well for low stickiness, and both show the symmetry
about $\rho_M = \frac{2}{3}$. For high stickiness, where the MFT prediction gives negative diffusion constant, the simulation generates a much increased density in the system.  This takes $\rho_M$ outside the negative-$D$ regime of the MFT,
and into slow, but positive $D$ regime; this might explain why our measured diffusion coefficients seem to have been ``stretched'' along the $\rho_M$ axis. As before, we don't see a consistent spiking in the variance of the flow rate,
which is generally consistent with out other findings with
regards to the accuracy of the MFT.
%Query

%The structure of the flow is shown in Fig 5.  We observe density correlations persisting over many sites and for very long timescales.  These high and low density structures show no systematic movement.
%Superimposed on the pattern we see narrow lines with well defined gradient (velocity) which we believe correspond to individual particles/vacancies diffusing through regions of low/high density.
It is instructive to get an overview of how the particles move during flow. Fig.~\ref{fig:flowPatterns} show a plot of the flow structure in an interesting regime. Additional plots can be found in the supplementary materials.

\begin{figure}[h!]
\caption{\label{fig:flowPatterns} Indicative spacetime flow pattern for sticky free-flow $\left[\lambda = \frac{3}{20}, (\rho_0, \rho_L) = (\frac{3}{4}, \frac{1}{4})\right]$; other combinations shown in the supplementary materials.
Time runs along the x-axis, space (1 pixel=1 site) along the y-axis, with grayscale tone illustrating average site occupation over (clockwise from top left) $\frac{1}{32}$, $1$, $8$ and $32$ Gillespie steps per site respectively.}
%\includegraphics[width=0.98\linewidth]{../tex-src/images/flowImps2/flowl2r2.png}
\begin{center}
 \begin{tabular}{c | c}
    \includegraphics[width=0.49\linewidth]{../tex-src/images/newFlowImps/shortTime}  &\includegraphics[width=0.49\linewidth]{../tex-src/images/newFlowImps/midShortTime} \\
    \hline
    \includegraphics[width=0.49\linewidth]{../tex-src/images/newFlowImps/longTime} &\includegraphics[width=0.49\linewidth]{../tex-src/images/newFlowImps/midLongTime}
    \end{tabular}
\end{center}
    \vspace{-2em}
\end{figure}

\iffalse
Fig.~\ref{fig:flowPatterns} shows the short-time-averaged local density as a function of space and time for a range of densities and stickiness.
\begin{figure*}[h!]
\caption{\label{fig:flowPatterns} The spacetime flow patterns, for the $(\lambda, \rho_M)$ combinations indicated in the row and column headers. In each plot time runs along the $x$-axis, space along the $y$-axis. White represents full occupation, black empty, and grey shades partial
occupation. The degree of occupation was calculated by taking the \texttt{KMCLib} record of a particular site's occupation (i.e. the Gillespie times at
which the site changed occupation), assigning $0$ and $1$ to particles and vacancies respectively, linearly interpolating this and then integrating over times longer than a single Gillespie step but much shorter than the total time in question.
In each case the total time elapsed is that taken by $10^6$ Gillespie steps, and each short-time-average has been done over the total time divided by $508$ (to produce square diagrams, as there are $508$ active sites
per simulation). Time has been rescaled this way in order to allow fair comparison of radically different $\lambda$-values.}
\begin{tabular}{c p{0.175\linewidth}}
\hspace{-2em}\begin{tabular}{c|c@{\hspace{0.25em}}c@{\hspace{0.25em}}c}
  &  $\lambda=0.05$ & $\lambda=0.38$ & $\lambda=1.00$ \\ 
  \hline
   \begin{tabular}{c} \vspace{-12em} \\ \hspace{-1em}$\rho_M=0.05$\hspace{0em} \\  \\ \end{tabular} & \includegraphics[width=0.32\linewidth]{../tex-src/images/flowImps2/flowl0r0.png} & \includegraphics[width=0.32\linewidth]{../tex-src/images/flowImps2/flowl2r0.png}  & \includegraphics[width=0.32\linewidth]{../tex-src/images/flowImps3/flowl0r0.png} \\
   \begin{tabular}{c} \vspace{-12em} \\ \hspace{-1em}$\rho_M=0.50$\hspace{0em} \\  \\ \end{tabular} & \includegraphics[width=0.32\linewidth]{../tex-src/images/flowImps2/flowl0r2.png} & \includegraphics[width=0.32\linewidth]{../tex-src/images/flowImps2/flowl2r2.png}  & \includegraphics[width=0.32\linewidth]{../tex-src/images/flowImps3/flowl0r2.png} \\
   \begin{tabular}{c} \vspace{-12em} \\ \hspace{-1em}$\rho_M=0.95$\hspace{0em} \\  \\ \end{tabular} & \includegraphics[width=0.32\linewidth]{../tex-src/images/flowImps2/flowl0r4.png} & \includegraphics[width=0.32\linewidth]{../tex-src/images/flowImps2/flowl2r4.png}  & \includegraphics[width=0.32\linewidth]{../tex-src/images/flowImps3/flowl0r4.png} \\
   \end{tabular}
\hspace{-1em}
&
\end{tabular}
\end{figure*}
When $\lambda$ is extremely low (left), the medium consists of solid blocks surrounded by empty spaces containing a dilute
gas of particles; as we alter the overall density, all that changes is the thicknesses of these blocks.   Thus the breakdown of MFT
is revealed as a decomposition into regions, of alternating low and high density, each of which allows similar flow rate.  The MFT solution, which assumed
intermediate density and gave negative diffusion constant, is revealed to be unstable.
The case $\lambda = 1$ (right) is just excluded Brownian motion, and is included here for comparison.
The most interesting images (centre) are those for the intermediate $(\rho_M , \lambda)$; here we see a ``lumpy'' or ``foamy'' structure, in which small blocks
of particles are being constantly created and destroyed whilst a rather minimal flow occurs across the system.
The simulations did not show any hard phase transition as we vary $(\rho_M, \lambda)$; rather, it seems that this ``foamy''
behaviour is part of a continuous range between the extremes, containing medium-range correlations between particles.
Unfortunately, computing equal-time correlation functions to the accuracy required
to draw conclusions about these correlations has proven to be extremely difficult, so we cannot find a quantitative description of the foam beyond the averages properties in Fig.~\ref{fig:lambdaScans}, \ref{fig:constDens},
and \ref{fig:diffCoef}.
In all images in Fig.~\ref{fig:flowPatterns}, long straight segments of white of black can be seen.  The represent coherent motion at a characteristic velocity given by their gradient. There is nothing in the MFT to suggest what this velocity
should be, and it is much smaller than the simulated system's length divided by the elapsed time,  $\frac{L}{T}$, thus it must be an emergent property arising from correlated motion of self-assembled regions of  high- or low-density material.

When $\lambda$ is extremely low, the medium consists of solid blocks surrounded by empty spaces containing a dilute gas of particles; as we alter the overall density,
all that changes is the thicknesses of these blocks. The case $\lambda=1$ is just excluded Brownian motion, and is included here for comparison. The most interesting images are those for the intermediate $(\rho_M , \lambda)$; here
we see a ``lumpy'' or ``foamy'' structure, in which small blocks of particles are being constantly created and destroyed whilst a rather minimal flow occurs across the system.
We do not think that there is any hard phase transition as we vary $(\rho_M , \lambda)$; rather, it seems that this ``foamy'' behaviour is part of a continuous range of phenomena between the extremes, containing medium-range correlations between
particles. However, numerically computing equal-time quantities such as the equal-time correlation function to the accuracy required to draw conclusions about these correlations has proven to be extremely difficult, so we cannot speak in quantitative terms about them.
\fi
\iffalse
\subsection{Correlation Functions}
Whilst we're calculating flow rates, we can also use our \texttt{KMCLib} code to calculate the equal time 2-point correlation function
$C(x) = \left\langle \rho(x)\rho(0) \right\rangle - \left\langle \rho(x) \right\rangle \left\langle \rho(0) \right\rangle $.
We can calculate the same quantity in a finite periodic ring analytically, and in both the analytic and numerical cases we may attempt to extract a correlation length by (curve-fitting / Laplace transform);
hence we can check whether having a steady flow through the system causes any structural effects.}
\fi


To conclude, we have solved a nonlinear model for self-interacting sticky particles diffusing in 1D.  Although only the particles exhibit stickiness,  the analytics suggest a symmetry between vacancy-type and particle-type flow
at density of $\frac{2}{3}$, which is observed in the simulation.  The flow exhibits a foamy pattern with intermediate time-and-space correlations.  The continuum solution MFT is a good predictor of the bulk flow behavior of the SPM.
The negative diffusion constant found in MFT at high stickiness indicates  that the assumption of homogeneous density break down: thus the MFT predicts its own demise.

We would like to thank EPSRC (student grant 1527137) and Wolfson Foundation for providing the funding, Mikael Leetmaa for producing \texttt{KMCLib}, and the \texttt{Eddie3} team here at Edinburgh for maintaining the hardware used.
We would also like to thank Martin Evans, Bartek Waclaw and Richard Blythe for some very helpful discussions during the production of this letter.
% Need to say something like `` See Supplemental Material at [URL will be inserted by publisher] for [give brief description of material]. ''

\bibliography{jHellSpring2017}



\iffalse

\section{\label{sec:level1}First-level heading}

This sample document demonstrates proper use of REV\TeX~4.1 (and
\LaTeXe) in mansucripts prepared for submission to APS
journals. Further information can be found in the REV\TeX~4.1
documentation included in the distribution or available at
\url{http://authors.aps.org/revtex4/}.

When commands are referred to in this example file, they are always
shown with their required arguments, using normal \TeX{} format. In
this format, \verb+#1+, \verb+#2+, etc. stand for required
author-supplied arguments to commands. For example, in
\verb+\section{#1}+ the \verb+#1+ stands for the title text of the
author's section heading, and in \verb+\title{#1}+ the \verb+#1+
stands for the title text of the paper.

Line breaks in section headings at all levels can be introduced using
\textbackslash\textbackslash. A blank input line tells \TeX\ that the
paragraph has ended. Note that top-level section headings are
automatically uppercased. If a specific letter or word should appear in
lowercase instead, you must escape it using \verb+\lowercase{#1}+ as
in the word ``via'' above.

\subsection{\label{sec:level2}Second-level heading: Formatting}

This file may be formatted in either the \texttt{preprint} or
\texttt{reprint} style. \texttt{reprint} format mimics final journal output. 
Either format may be used for submission purposes. \texttt{letter} sized paper should
be used when submitting to APS journals.

\subsubsection{Wide text (A level-3 head)}
The \texttt{widetext} environment will make the text the width of the
full page, as on page~\pageref{eq:wideeq}. (Note the use the
\verb+\pageref{#1}+ command to refer to the page number.) 
\paragraph{Note (Fourth-level head is run in)}
The width-changing commands only take effect in two-column formatting. 
There is no effect if text is in a single column.

\subsection{\label{sec:citeref}Citations and References}
A citation in text uses the command \verb+\cite{#1}+ or
\verb+\onlinecite{#1}+ and refers to an entry in the bibliography. 
An entry in the bibliography is a reference to another document.

\subsubsection{Citations}
Because REV\TeX\ uses the \verb+natbib+ package of Patrick Daly, 
the entire repertoire of commands in that package are available for your document;
see the \verb+natbib+ documentation for further details. Please note that
REV\TeX\ requires version 8.31a or later of \verb+natbib+.

\paragraph{Syntax}
The argument of \verb+\cite+ may be a single \emph{key}, 
or may consist of a comma-separated list of keys.
The citation \emph{key} may contain 
letters, numbers, the dash (-) character, or the period (.) character. 
New with natbib 8.3 is an extension to the syntax that allows for 
a star (*) form and two optional arguments on the citation key itself.
The syntax of the \verb+\cite+ command is thus (informally stated)
\begin{quotation}\flushleft\leftskip1em
\verb+\cite+ \verb+{+ \emph{key} \verb+}+, or\\
\verb+\cite+ \verb+{+ \emph{optarg+key} \verb+}+, or\\
\verb+\cite+ \verb+{+ \emph{optarg+key} \verb+,+ \emph{optarg+key}\ldots \verb+}+,
\end{quotation}\noindent
where \emph{optarg+key} signifies 
\begin{quotation}\flushleft\leftskip1em
\emph{key}, or\\
\texttt{*}\emph{key}, or\\
\texttt{[}\emph{pre}\texttt{]}\emph{key}, or\\
\texttt{[}\emph{pre}\texttt{]}\texttt{[}\emph{post}\texttt{]}\emph{key}, or even\\
\texttt{*}\texttt{[}\emph{pre}\texttt{]}\texttt{[}\emph{post}\texttt{]}\emph{key}.
\end{quotation}\noindent
where \emph{pre} and \emph{post} is whatever text you wish to place 
at the beginning and end, respectively, of the bibliographic reference
(see Ref.~[\onlinecite{witten2001}] and the two under Ref.~[\onlinecite{feyn54}]).
(Keep in mind that no automatic space or punctuation is applied.)
It is highly recommended that you put the entire \emph{pre} or \emph{post} portion 
within its own set of braces, for example: 
\verb+\cite+ \verb+{+ \texttt{[} \verb+{+\emph{text}\verb+}+\texttt{]}\emph{key}\verb+}+.
The extra set of braces will keep \LaTeX\ out of trouble if your \emph{text} contains the comma (,) character.

The star (*) modifier to the \emph{key} signifies that the reference is to be 
merged with the previous reference into a single bibliographic entry, 
a common idiom in APS and AIP articles (see below, Ref.~[\onlinecite{epr}]). 
When references are merged in this way, they are separated by a semicolon instead of 
the period (full stop) that would otherwise appear.

\paragraph{Eliding repeated information}
When a reference is merged, some of its fields may be elided: for example, 
when the author matches that of the previous reference, it is omitted. 
If both author and journal match, both are omitted.
If the journal matches, but the author does not, the journal is replaced by \emph{ibid.},
as exemplified by Ref.~[\onlinecite{epr}]. 
These rules embody common editorial practice in APS and AIP journals and will only
be in effect if the markup features of the APS and AIP Bib\TeX\ styles is employed.

\paragraph{The options of the cite command itself}
Please note that optional arguments to the \emph{key} change the reference in the bibliography, 
not the citation in the body of the document. 
For the latter, use the optional arguments of the \verb+\cite+ command itself:
\verb+\cite+ \texttt{*}\allowbreak
\texttt{[}\emph{pre-cite}\texttt{]}\allowbreak
\texttt{[}\emph{post-cite}\texttt{]}\allowbreak
\verb+{+\emph{key-list}\verb+}+.

\fi

\end{document}
%
% ****** End of file apssamp.tex ******
