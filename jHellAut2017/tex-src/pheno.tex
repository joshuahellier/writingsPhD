The model described in Fig.~\ref{fig:rates} is very simple, but numerical simulation shows that it is capable of a wide range of behaviors, such as those shown in Fig.~\ref{fig:flowPatterns}. We will discuss
these numerical results in more detail later, but first let us analyze the model behavior using analytic means.
Because this model contains interactions, the types of methods for the full analytic solution of SEP don't help us; thus, the best we can currently do is a mean field theory approximation.
Let the spacing between lattice sites be $a$, let $\tau_0$ be the free-particle hopping timescale, and the time-averaged (or ensemble-averaged, assuming ergodicity) occupation probability of the $i^{\mathrm{th}}$ lattice site be $\rho_i$.
We introduce $\zeta = 1 - \lambda $ here for convenience.
One may show that, in the mean-field approximation regime,
% maybe derive in appendix
\begin{align}
\begin{split}
 \tau_0 \partDeriv{\rho_i}{t} = &\left( 1-\rho_i \right) \left[ \left(1-\zeta\rho_{i-2} \right) \rho_{i-1} + \left(1-\zeta\rho_{i+2} \right) \rho_{i+1} \right] \\
 &- \rho_i \left[ 2 \zeta \rho_{i-1} \rho_{i+1}  - (3-\zeta)\left(\rho_{i-1} + \rho_{i+1}\right) + 2 \right].
 \end{split}
 \end{align}
Switching to the continuum limit by taking $a\rightarrow 0$, and neglecting $\mathcal{O}(a^4)$ terms, we may re-express this as a conserved flow $J$ as follows:
\begin{align}
 \partDeriv{\rho}{t} &= - \partDeriv{J}{x}, \\
 J &= -  D(\rho) \partDeriv{\rho}{x}, \\
 D(\rho) &= \frac{a^2}{\tau_0} \left[1 - \zeta \rho\left(4-3\rho\right) \right]. 
\end{align}
Thus, the MFT says that the particles should diffuse with a diffusion coefficient $A(\rho)$ which depends upon the local density.

In order to understand the implications of the MFT, let us consider some limits. As $\zeta \rightarrow 0$ (i.e. as the model becomes a simple exclusion model), $D \rightarrow \frac{a^2}{\tau_0}$. Likewise, in the
dilute limit $\rho \rightarrow 0$, $D \rightarrow \frac{ a^2}{\tau_0}$, reflecting the fact that it becomes a dilute lattice gas and therefore the interactions between particles become irrelevant as they never meet.
Conversely, in the full limit $\rho \rightarrow 1$, $D \rightarrow \frac{\lambda a^2}{\tau_0}$; this is because we now have a dilute gas of vacancies, which hop with rate $\frac{\lambda}{\tau_0}$.
One may observe that the continuum limit MFT has a symmetry under $\rho \mapsto \frac{4}{3} - \rho$; thus, the dynamics should be symmetric under a density profile reflection around $\rho = \frac{2}{3}$. This is where $D$ always
attains its extremal value, $ \frac{a^2}{\tau_0}\left[1 - \frac{4}{3}\zeta\right]$, hence for $\zeta>3/4$ the diffusion coefficient becomes negative in regions with
$\frac{2}{3} - \frac{\sqrt{\zeta\left(4\zeta - 3\right)}}{3\zeta} < \rho < \frac{2}{3} + \frac{\sqrt{\zeta\left(4\zeta - 3\right)}}{3\zeta}$.
Finally, it is possible to show that solutions to the continuum MFT containing domains with a negative diffusion coefficient are linearly unstable; thus, if we try to have a flow containing $\rho$ for which $D(\rho)<0$,
the density of the medium should gravitate towards a density for which $D(\rho)\sim 0$. Instead of observing ``backwards diffusion'' we would see an extremely slow flow or no flow at all. The MFT implies that the transition
to this critically slowly-flowing regime happens suddenly, like a phase transition: this can be checked with our numerics.

It is possible to solve the continuum MFT in a steady state on a finite domain, say $x\in(0, L)$. The continuity equation implies that $J(x)=J_0$, and by integrating both sides of that equation with respect to $x$ we find that
\begin{equation}
 J_0 (x-x_0) = -\frac{a^2}{\tau_0} \rho \left[1+\zeta \rho\left(\rho-2\right)\right],
\end{equation}
a cubic equation which can be solved to give $\rho(x)$. If we impose Dirichlet boundary conditions on this system, say $\rho(0)=\rho_0$ and $\rho(L)=\rho_L$, we find that
\begin{equation}
 J = \frac{a^2}{L \tau_0} \left[ \rho_0 - \rho_L + \zeta \left( \rho_0\left[\rho_0^2-2\right] - \rho_L\left[\rho_L^2-2\right] \right) \right].
\end{equation}
We may consider applying small concentration gradients across a block by setting $\rho_0 = \rho_M + \frac{1}{2}\delta\rho$ and $\rho_L = \rho_M - \frac{1}{2}\delta\rho$. Doing so, we find that the effective diffusion coefficient of the block
$D_\mathrm{Eff}=\partDeriv{J}{\delta\rho}\big|_{\delta\rho=0}$ obeys
\begin{equation}
\label{eq:MFTflow}
 \partDeriv{J}{\delta\rho}\bigg|_{\delta\rho=0} = \frac{a^2}{L \tau_0} \left[ 1 - \zeta\rho_M(4-3\rho_M) \right].
\end{equation}