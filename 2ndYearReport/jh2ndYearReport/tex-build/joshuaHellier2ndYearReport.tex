%
%                       This is a basic LeTeX Template
%                       for the First Year PhD literature review 
\documentclass[a4paper,12pt]{article}
\usepackage{head,fullpage,epsf} % Add local fullpage and eps macros
\usepackage{graphicx}
\usepackage{amsmath}
\usepackage{amssymb}
\parindent=0pt          %  Switch off indent of paragraphs 
\parskip=5pt            %  Put 5pt between each paragraph  
%
%                       This section generates a title page
%                       Edit only the sections indicated to put
%                       in the project title, your name, supervisor,
%                       project length in weeks and submission date
%
\newcommand{\partDeriv}[2]{\frac{\partial #1}{\partial #2}}
\DeclareMathOperator\erf{erf}

\begin{document}
\epsfxsize=40mm                         % Size of crest
\begin{minipage}[b]{110mm}
        {\Huge\bf School of Physics \\and Astronomy
        \vspace*{17mm}}
\end{minipage}
\hfill
\begin{minipage}[t]{40mm}               
        \makebox[40mm]{
        \includegraphics[width=\linewidth]{../tex-src/images/crest-crop}}
\end{minipage}
\par\noindent                                           % Centre Title, and name
\vspace*{2cm}
\begin{center}
        \Large\bf Interacting Diffusion in Metals\\
        \Large\bf Second Year Report
\end{center}
\vspace*{1.5cm}
\begin{center}
        \bf Joshua DM Hellier\\                 % Replace with your name
        June 2016                         % Submission Date
\end{center}
\vspace*{5mm}
%
%                       Insert your abstract HERE
%                       
\begin{abstract}
Corrosion is an important process, which may severely limit the utility of an alloy. In this PhD project, I am currently developing a simple 
model of the way Oxygen diffuses through pure Titanium, in an effort to capture the saliant features of this phenomenon. Once I am satisfied this model works, I wish to extend it to Titanium alloys
in order to explain why some of them are better at resisting corrosion than others.
\end{abstract}

\vspace*{1cm}

\vspace*{3cm}
Signature:\hspace*{8cm}Date:

\vfill
{\bf Supervisor:} Professor Graeme Ackland                % Change to suit
\newpage
%                                               Through page and setup 
%                                               fancy headings
\setcounter{page}{1}                            % Set page number to 1
\footruleheight{1pt}
\headruleheight{1pt}
\lfoot{\small School of Physics and Astronomy}
\lhead{Second Year Report}
\rhead{\thepage}
\cfoot{}
\rfoot{Date: \today}
%
\tableofcontents  % Makes Table of Contents

\section{Introduction}

I apologise for the lateness of submission of this document; I was trying to get a little more research done before submitting it, and indeed I did derive some of the formulae about lattice gas equilibria
after the submission deadline. Throughout this report I will assume that the reader has already read my First Year Report, so I won't be rereferencing anything from that.

\section{Progress Since First Year Report}
\subsection{Solutions to the 1-D Transport Equation in Moving Frames}

The experimental evidence presented in~\cite{tegner2015high} suggests that the interface growth rate in Nb-alloyed Titanium is sublinear; therefore, the process is likely to be limited by diffusion. Thus, I briefly investigated the 1-D diffusion equation in frames moving with time according to a power law, which takes the form
\begin{equation}
\partDeriv{u}{t} - n \nu t^{n-1} \partDeriv{u}{x} = \kappa \partDeriv{^2 u}{x^2}.
\end{equation}
Here, the frame is moving as $x = t^n$, for some real $n$ and hence we have used the convective derivative $ \partDeriv{u}{t} - n \nu t^{n-1} \partDeriv{u}{x} $ to take account of this motion. My reason for doing this is to allow us to easily
follow an interface which is being pushed forward by a diffusive process, as
such an interface would be described by $x$ being held constant (in particular at 0). 

In the spirit of the Fluid Dynamicists~\cite{batchelor2000introduction}, let us first switch from $(x, t, u)$ to a set of dimensionless parameters, $(\xi, \eta, u)$~\footnote{Note that the dimension of u is immaterial, as it appears once in every term.}.
We see straight away that the variable $\xi = \frac{x}{\sqrt{\kappa t}}$ is dimensionless, and indeed this is the same dimensionless quantity encountered when attempting similarity solution of the standard 1-D diffusion equation. In the moving frame, there is an additional dimensionless variable $\eta = \frac{x}{\nu t^n}$. In general, we would now transform into the new coordinates and seek $u(\xi, \eta)$ as the solution to the new PDE.

However, the case $n=\frac{1}{2}$ (which happens to be the case we are most interested in, as it represents sublinear ``quadratic growth'') is special, as $\xi$ and $\eta$ in this case are not independent; indeed, we can see that $\nu^2 \eta^2 = \kappa \xi^2$. Thus, in the situation that there is no particular scale being imposed upon the system by the initial data,  any solution may now be written as $u(\xi)$. Under these assumptions, the PDE reduces to the ODE
\begin{equation}
u'' + \frac{1}{2}\left( \xi + \nu \kappa^{-\frac{1}{2}} \right)u = 0
\end{equation}
where $'$ denotes differentiation with respect to $\xi$. This has the general solution
\begin{equation}
u(\xi) = A \erf{\left[\frac{1}{2}\left(\xi + \frac{\nu}{\sqrt{\kappa}}\right)\right]} + B
\end{equation}
for $A$, $B$ arbitrary real constants. Substituting in our definition of $\xi$, we see that the solution takes the form
\begin{equation}
u(x, t) = A\erf{\left[\frac{1}{2} \left( \frac{x+\nu \sqrt{t}}{\kappa \sqrt{t}} \right)\right]} + B
\end{equation}
It is interesting to note that the particle current,
\begin{equation}
J = -\kappa - \frac{1}{2} \nu t^{-\frac{1}{2}}
\end{equation}
passing through $x=0$ varies in proportion to $t^{-\frac{1}{2}}$, which happens 
to be the rate of deposition required to build structure in quadratic growth.
I intended to use this solution to build an Oxygen/Antioxygen model for the TiO$_2$/Ti interface, but have put this on hold as it would require detailed knowledge of the interface region, which presents quite a challenge; I will 
probably return to this question later, possibly armed with computational 
results. Meanwhile, I have recently turned my attentions to the diffusion of
dilute Oxygen/Niobium mixtures in Titanium, which is the subject of the next 
section.

\subsection{Deriving a Simple Interacting Multi-Species Diffusion Model}







\section{Courses, Publications, and Plans}
Given that \texttt{KMCLib} has a Python frontend, a language with which I was not familiar, I took SUPA Python course this year. Other than that, I have attended no other courses, since, including that,
I believe I have already met all of my SUPA requirements.

As of yet, I have no publications. However, I would like to be ready to publish on the subject of the phenomenology of the one-dimensional lattice model described in~\ref{sec: modelChoice} by the end of the Summer.
In order to do this, I have the following short-term goals:
\begin{enumerate}
 \item Get the code working properly and stably. The input files have become a bit of a mess as I have experimented with them and need to be cleaned. Also the computations are currently memory-limited, which is not a great situation
 from a scaling-up perspective, so I need to develop cleverer ways of taking, storing and processing results.
 \item I need to work out how to port the code to some cluster in order to perform large-scale calculations; currently I am just running it on my laptop. I believe \texttt{Eddie3} would be a good choice of machine. In order to
 be able to run on such a system, I will need to tweak the code further.
 \item Once I have the capability to do lots of calculations fairly fast, I would like to firmly establish what the system does in equilibrium (as in, does it obey equation~\ref{eq: eqmBehaviour}?), and then move onto investigating dynamical
 behaviour in which Oxygen is driven through the system via concentration gradients. At that point we can compare to the mean-field long-wavelength PDE in equation~\ref{eq: longRangePDE}.
\end{enumerate}
Once I have done these things, I think that will be enough to publish in a paper.

I'm not entirely sure what to do after that; I will probably be in a better position to plan my next moves once this is done. Certainly I would like to extend the model to include alloying atoms such as Nickel, and it would also
be good to extend the model to three-dimensions, to make sure the behaviour isn't that different from that observed in one-dimension. I would like to have at least made the Nickel extension by the end of my PhD.


%
%                       Here is how to inserted a centered
%                       postscript file, this one is actually
%                       out of Maple, but it will work for other
%                       figures out of Xfig, Idraw and Xgraph
%
%%
%\begin{figure}[htb]     %Insert a figure as soon as possible
%        \begin{center}
%          \epsfxsize=100mm         % Horizontal size YOUR want
%          \epsffile{OpticalSystem.eps}
%\end{center}
%\caption{Here is the optical system from the same paper
%  as the reference drawn in {\tt xfig} and include to
%  shown how such a figure in included.}
%\label{fig:prism}                 % Reference label to the figure.
%\end{figure}
%%

 

%            Now build the reference list
\bibliographystyle{unsrt}                      % The reference style
%                This is plain and unsorted, so in the order
%                they appear in the document.


\bibliography{writeup}       % Multiple bib files.

\end{document}

