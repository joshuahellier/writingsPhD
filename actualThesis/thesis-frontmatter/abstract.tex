\chapter{Abstract}

Diffusion processes are pretty ubiquitous across the natural world, so it is important to try to understand them. A system in which diffusion is being driven by concentration differences between boundary reservoirs is a simple example of a nonequilibrium statistical mechanics system. In this thesis, we study a model which has been hanging around the literature in one form or another for a long time; the Sticky Particle Model, or SPM. This is a very basic one-parameter exclusion model, in which particles move away from adjacent particles with a different rate to their normal free movement. We use a variety of techniques to analyse induced flow in this model, including a simple analytic mean-field theory, Monte Carlo calculations, as well as direct numerical analysis of the transition rate operator which corresponds to small versions of the system. During these investigations, we have discovered what we argue is a nonequilibrium phase transition between flow regimes at high and low stickiness values of our “stickiness parameter”; much of our work has gone into attempts to understand the nature of this apparent transition.
