\chapter{Conclusions} \label{sec:conclusionsChap}
During the course of this research, we have used a whole battery of methods in order to try to understand
the behaviour of the Sticky Particle Model. Here we present our main conclusions, and suggest topics which
future researchers may wish to investigate.

Our main conclusions about the Sticky Particle Model are as follows:
\begin{itemize}
 \item The Sticky Particle Model, and its $n$-dimensional generalisation, is the unique stochastic exclusion 
 model defined on a square $n$-dimensional lattice in which the particle dynamics are completely symmetric,
 depend only upon a particle's immediate environment and obey detailed balance (Sec.~\ref{sec:uniqueProof}).
 \item The partition function and chemical potential for the SPM on a closed ring can be calculated analytically (Sec.~\ref{sec:spmPartFn}).
 \item The $n$-dimensional SPM has simple continuum-limit mean-field approximation which can be easily
 calculated (Sec.~\ref{sec:nDMftDeriv}). It isn't great at predicting behaviour in $1$D, but in higher dimensions
 it is quite good.
 \item In $1$D we can use our numerical Transition Rate Matrix method to exactly solve, to great accuracy,
 small SPM systems (Chap.~\ref{sec:transRateChapter}).
 \item Using TRM methods, we find that in $1$D we observe a \textit{power-law switching} phenomenon, whereby the dependence of the boundary-induced current upon the stickiness parameter $\lambda$ switches from
 varying linearly with $\lambda$ to varying approximately cubically with $\lambda$ 
 (Sec.~\ref{sec:TRMDensityCurrent}). Whilst this does not appear to be a discontinuous phase transition in the
 normal sense, the system does undergo rapid change when $\lambda$ is varied only slightly. We did not find
 sufficient evidence of the types of behaviour which usually accompany phase transitions (e.g. discontinuous order
 parameters, diverging fluctuation sizes, etc), we do not believe we can call this change a phase transition;
 however, it might be reasonable to refer to the continuous change in  behaviour from one regime to another as a
 \textbf{crossover}.
 \item Our TRM observations in $1$D are broadly confirmed by larger-scale Monte-Carlo simulations 
 (Sec.~\ref{sec:1dMonteCarlo}).
 \item In $2$D our Monte-Carlo calculations have revealed an interesting phenomenon whereby there appear
 to be multiple stable (or possibly metastable) states which can occur in small SPM systems with given boundary
 conditions at small $\lambda$ (Sec.~\ref{sec:2dLambdaScans}); these states may be identified by their 
 different densities and current dependencies.
\end{itemize}

We have also developed many codes and other technologies in order to carry out our investigations~\cite{hellier2019a, hellier2019b}. Of the methods we have developed, we are perhaps most proud of the TRM method, which
isn't something we have seen the like of in the literature. We suspect that it could be put to good use on a
variety of other low-dimensional systems, to give researchers insight into the behaviour of these systems on 
small scales. This small-scale behaviour can often be a good window into larger-scale phenomena, and using TRM
analysis one might pick up on things to calculate that one wouldn't usually think to investigate.

In terms of future work to be done in this area, we would suggest that other systems might exhibit
power-law switching, and so this should be checked. We would also suggest that the community
might wish to look again at models which are symmetric and obey detailed balance; although their
equilibrium properties in closed boundaries are often rather trivial, this work has shown that they are 
capable of doing interesting things when driven by boundary conditions. Finally, we think that the
odd hysteresis-like phenomenon we saw in $2$D is well worth further investigation, and regret not having the
time to pursue it further.


