\chapter{Monte-Carlo Simulations of the SPM} 
\label{sec:numerics}
We now have numerical results for SPM systems using TRM analysis; however, this only allows us to study
relatively small systems. In order to study larger ones, we have used Monte-Carlo methods. In this
chapter, we will discuss the methods we used, the results they yielded and their meaning, with 
particular emphasis on what they tell us about the suspected transition between low and high-$\lambda$
behaviours.
\section{Numerical Simulations of Continuous-Time Markov Processes}
\subsection{Purpose of Monte-Carlo Methods}
We should first really describe what we mean by a Monte-Carlo method. In essence, Monte-Carlo methods
refer to numerical routines in which we attempt to characterise an unknown distribution by using
pseudorandom numbers in order to produce sample data which is hopefully faithful to the original 
distribution, at least in terms of the statistics we are trying to calculate. A good example of a
commonly-used Monte-Carlo method in Physics is the Metropolis-Hastings algorithm, which in its original
for is used to calculate statistics for equilibrium statistical mechanics systems.

In our situation, we wish to be able to mimic a continuous-time discrete-state Markov process.
As we saw in Chap.~\ref{sec:TRM}, the state space for a TRM system of size $L$ scales as
$\mathcal{O}(2^L$); thus we quickly run out of size if we try to consider exactly probability distributions, which correspond to
vectors in $\mathbb{R}^{L^2}$. We can, however, store individual configurations, which only occupy
$\mathcal{O}(L)$ space. Therefore, we need to find a way to produce trajectories through the discrete
state space which sample the actual space of system trajectories well enough to allow us to access the
statistics we want. Of course, there isn't a unique ``best'' way to do this. We have considered two
contrasting methods, which differ primarily in the way in which they convert the original continuous 
time into discrete steps which we can use in an algorithm.

\subsection{Evenly-Spaced Timesteps}
\subsection{The N-Fold Way, or Gillespie Algorithm}
\section{Implementation of Monte Carlo Methods}
\subsection{Our Implementation of a Metropolis-Hastings}
\subsection{\texttt{KMCLib}}
Talk about how it works, why I picked it over other implementations.
\subsection{Running our Monte-Carlo Calculations}
How calculations are managed day-to-day.

\section{1D Calculation Results}
\subsection{Calculational Choices}
\subsection{Flow Patterns}
\subsection{Current}
\subsection{Density}
\subsection{Diffusion Coefficient}

\section{2D Calculation Results}
\subsection{Calculational Choices}
\subsection{Results}

\section{Conclusions}
