\chapter{Transition Rate Matrix Analysis} 
\label{sec:transRateChapter}
Now that we have MFT predictions about the relationship between density difference and
current in the SPM, it would be good to try to investigate their validity. In Chapter~\ref{sec:numerics} we will use Monte-Carlo methods to do this in $1$d and $2$d, but in this chapter we will restrict our attention to $1$d.

\section{The Transition Rate Operator for the SPM}
The SPM is a continuous-time Markov Process, which describes continual transitions between states
with transition rates depending only upon the current state. As such, if we call the total space of
states $\Xi$ then the probability distribution $P: \Xi \times \mathbb{R} \rightarrow  \mathbb{R}$ should obey a \textbf{master equation}
\begin{equation}
 \partDeriv{P(\xi, t)}{t} = \mathcal{A} P(\xi, t),
\end{equation}
where $\mathcal{A}:\Xi \rightarrow \Xi$ is the \textbf{transition rate operator}
or TRO. Parametrising $\mathcal{A}$ via 
\begin{equation}
 (\mathcal{A}f)(u) = \int_\Xi \! \! \mathrm{d}  \xi \ \sigma (u, \xi) f(\xi)
\end{equation}
puts it in a more familiar, cross-section notation. To conserve finite probability mass
we demand that $\sigma$ satisfies
\begin{equation}
 \forall \xi \in \Xi, \sigma (\xi , \xi) \le 0 
\end{equation}
and
\begin{equation}
 \forall \xi_1 , \xi_2 \in \Xi : \xi_1 \ne \xi_2 \sigma (\xi_1 , \xi_2) \ge 0 ,
\end{equation}
as well as the constraint
\begin{equation}
 
\end{equation}


\section{Forming the TRM for Finite Systems}
About the algorithm to construct the sparse TRM; how it's formed and all that stuff.

\section{The Eigenspectrum of the TRM}

\subsection{The Computation of the TRM Eigenspectrum}

\subsection{The Structure of the TRM Eigenspectrum}

\subsection{Current and Density in the Steady State}

\section{Time-Dependent Properties of Small SPM Systems}

\subsection{The Relaxation Time for the SPM}

\subsection{Autocorrelation Functions for the TRM}

\section{Conclusions}
