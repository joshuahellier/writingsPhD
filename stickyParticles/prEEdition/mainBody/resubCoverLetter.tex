\documentclass[a4paper,10pt]{article}
\usepackage[utf8]{inputenc}

%opening
\title{}
\author{Joshua DM Hellier and Graeme J Ackland}

\begin{document}

\maketitle

 Dear PRX Editors, \\ 

We would like to submit the paper ``On The Diffusion of Sticky
Particles in 1-D'' for consideration in PRX.  In it we create a new
paradigm for modelling complex flow, combining the simplest
particle-based model for flow with the simplest particle-based model
for attraction, the Symmetric Exclusion Process and the Ising model.
A remarkable discovery is that this ``Sticky Particle Model'' has a
nonequilibrium transition between two types of flow, something 
generally believed impossible in 1D without long-range interactions or
an external field.  As with the SEP and Ising models, the simple
nature of the model makes it applicable as a paradigm for a wide range
of flow problems.  We also demonstrate well-defined limiting behaviour
of the density and flow.  Simple explanations of these observations
escape us, but should open up the field to follow-up test and
developments of overarching theories on nonequilibrium thermodynamics
such as maximum entropy, maximum entropy production or maximum flow.

After introducing the model, we analyse it in depth with three
different techniques: Monte Carlo simulation, Transition rate
matrices, and mean field theory.

The paper has some history.  We originally submitted to PRL where the
referees felt we hadn't proved our case.  We are confident in our
results, but the refereeing at PRL made it clear that attempting to
cram both simulation and mean field theory into a letter format could
not be done in a clear manner.

Since the PRL submission, we developed the transition matrix approach
which provides yet more evidence and explanation of the model.  We are
confident that this, and extended numerical evidence of the transition
such as the fluctuation peak in 7 would convince the referees.  For
this reason, we choose to submit our more thorough exposition to
PRX. 



\end{document}


 Thank you for the response of the referees. While the responses are
fair, we feel the referees have looked primarily at the more trivial
responses, and that Referee A, in particular, has misunderstood the
main result. We recognise that the failure of experts to see the key
results means we failed to emphasize them appropriately, perhaps due
to our overfamilirity from working on the problem for several
years. Consequently, we have thoroughly reworked the manuscript, and
we would like to resubmit for consideration by PRL.

\subsection*{General Comments}
  Referee A states:
{\it  Since the main result of the paper seems to be a consequence of a ``bad'' approximation --- namely a mean-field approximation ---, rather than a description of the actual phenomenology of the model, I do not recommend to publish this manuscript in PRL.}

  This is mistaken: the main result is about the transition being in the actual phenomenology of the model, as measured numerically, so the referee's central reason for rejection does not hold.
 
  Referee B, by contrast, appears to think it obvious that ``the actual phenomenology of the model'' , has a transition, but is unconvinced that the details are sufficiently well presented to be a significant advance.

 While accepting that ``significance'' is always a matter of opinion,
 the referee's comments seem to focus on the necessary, but less
 significant, results of the work.
 
We would like to mention that, in our opinion, the analytic derivation
of the mean field solution (eq. 1, derived in the SM) is a significant
achievement and the difficulty of doing it is probably why this
apparently-obvious model has not been previously published.


In our revised version, we have addressed their comments and the paper is enormously improved in consequence.
 
 
\subsection*{Detailed response}


   \subsubsection*{Referee A}
  
   {\it This work presents a mean-field study of a one-dimensional non
   equilibrium transport model: particles diffuse on a lattice, and
   interact through both excluded-volume interactions and through
   attractive nearest-neighbor interactions. Mean-field analysis shows a
   phase transition from a regime with positive flux to a regime without
   flux.
  
   This manuscript has one issue: the main result found, namely a phase
   transition from a positive-flux phase to a zero-flux phase, seems to
   be an artifact of the mean-field approximation (see figure 2 and
   figure 4). At least, that is what the authors conclude: "However, one
   of the key predictions of the MFT - that a sharp transition to a
   no-flow regime occurs when ``$\lambda$'' becomes small enough (at least
   for 3 of the 4 sets of boundary conditions we investigated here) - is
   not realized in our simulations."
  
   The fact that the main result is only recovered in a mean-field
   approximation is not clearly mentioned in the abstract, where it is
   stated that: ``Neither the Ising model nor SEP have a phase transition
   in 1D, but the SPM exhibits a non-equilibrium transition from a
   diffusive to a blocked state as stickiness increases.'' Can the authors
   clarify: does the model has a non equilibrium phase transition, or
   does the model has no non-equilibrium phase transition?
  
   Since the main result of the paper seems to be a consequence of a
   ``bad'' approximation -namely a mean-field approximation-, rather
   than a description of the actual phenomenology of the model, I do not
   recommend to publish this manuscript in PRL.}




  The referee asks us to clarify the main result: ``does the model have a non equilibrium phase transition?''. 
  It does.
 
  Our simulations clear show the non-equilibrium transition is to a
  non-diffusing state.  The existence of the transition is not just an
  artefact of the MFT.
 
  We failed to make this main point clearly enough.
  Therefore we have made significant changes to the paper: in particular the
  new Fig.~1 is designed to emphasize the point, by showing the
  transition both according to MFT and simulation in the same plot.
  The previous figure, which failed to convince the referee, is
  included as an inset.  The main figure extends the range of the
  stickiness parameter to show more clearly the excellent agreement of
  MFT and simulation at the ``non-sticky'', high lambda regime, the
  transition of MFT to negative flow, and the equivalent transition to
  the blocked state in the real system.
 
  As for our MFT, we never expected it particularly good predictor in the very sticky limit. However, it is the breakdown of the MFT which suggested that the transition might occur in the first place,
  and inspired us to look for the transition, which we found.
 

The statement that {\it the sharp transition is not realised in the
  simulation} was intended to refer to the general result that finite size effects {\bf
  always} preclude sharp divergencies.  We now realise how this could
be misinterpreted and have removed it.
 
  
{\it    Minor comments:
  
   1) In the abstract it is said that:
  
   ``We prove that SPM is also a detailed-balance respecting,
   particle-conserving, Monte Carlo description of the Ising model.''
   This sentence seems to indicate that the authors prove an outstanding problem. But, isn't the detail balance criterion an immediate consequence of the definition of the model? }
 
   Of course, detailed balance (or not) is always a consequence of the
   definition of a model.  However, we did not deliberately set it up
   to obey detailed balance so, at least to us, detailed balance
   wasn't immediately obvious. 
  
   The model is defined in terms of rates, and because we did not
   notice it at first, we feel that detailed balance is not trivial
   and worth proving (see Referee B point 1). Most well-known jamming
   models defined by rates, e.g. ASEP, do not have detailed balance.
   It is sometimes tacitly assumed that this lack of detailed balance is
   necessary for non-trivial flow behaviour.  An important aspect of
   this work is to demonstrate that obtaining non-trivial flow *can*
   be achieved in a model with detailed balance.
 
So the referee is right to say that the proof of detailed balance is
not an outstanding achievement of the paper, but it is important to
establish it for contrast with apparently related work.

{\it 
   2) ``Thus, the MFT says that the particles should diffuse with a diffusion coefficient $A(\rho)$ which depends upon the local density.''
  
  
   I think $A(\rho)$ should be $D(\rho)$.}

The referee is correct, this typo has been corrected.

\subsubsection*{Referee B}
{\it
   In the manuscript ``On The Diffusion of Sticky Particles in 1-D'', the
   authors look into a variant of the so-called Symmetric Exclusion
   Process (SEP). They consider a 1d lattice model in which lattices
   sites can be occupied or empty, and particles may hop to empty
   adjacent sites. The difference with the usual SEP model stems from the
   considered transition rate, which in this work depends on the state of
   the hopping particle's nearest neighbors sites. If one of these sites
   is occupied, the hopping rate is reduced by a factor $\lambda$ and
   this is reasoned to be modelling ``stickiness'' between neighboring
   particles. This model, although very simple, does not seem to have
   been proposed before in the physics literature, although see more
   concrete comments on this point below.
  
   My reading of the paper ended with mixed feelings, so to say. On the
   one hand, I think that the results, which are interesting and
   non-trivial, reported in this paper are solid and thus the paper is
   undoubtedly publishable science. On the other hand, the paper is not
   well written and, without doubt, the results can be presented in a
   much clearer way. Besides, the paper lacks a discussion on the
   relation of the present work with other approaches, such as the
   general field of dynamical 1d Ising models or the diffusion of sticky
   particles in 1d. With regard to the specific criteria for acceptance
   in the Physical Review Letters, honestly I do not see that this paper
   meets any of them: the work is certainly interesting to the community
   of people working in 1d systems, who look for
   exact/perturbative/approximate results therein, but not for all
   physicists. In this particular field, I would not say that this work
   represents a substantial advance; after a major revision the obtained
   results might certainly be published in a specialized journal such as
   the Physical Review E (or JSTAT or JSP, thinking of journals outside
   the APS ecosystem).
}

We feel that the referee has missed the more important contributions
made by the paper.  In particular that the model is locally symmetric,
homogeneous, and obeys detailed balance yet {\bf still} shows jamming
behaviour.

The referee states that {\it the paper is not well
  written and, without doubt, the results can be presented in a much
  clearer way}.  

This echoes the criticism of Referee A, and perhaps explains the
referee's mixed feelings.  PErhaps we overemphasized technical details
at the expense of the bigger picture. We have restructured the paper
to emphasize more clearly those contributions we felt were missed by
the referees.

{\it

   Some more specific points below:
  
   1) The authors should enhance the part of the paper devoted to the
   introduction of the model and link it with other approaches, such as
   dynamical Ising models or sticky particles in 1d.
}
We have done this, including the connection to Kawasaki dynamics in the Ising
model and the links to SEP, Ising, KLS, misanthrope and KPZ models.

{\it  
   a) The authors state that ``It seems that space of highly-local
   exclusion models is so tightly constrained in one-dimension that there
   is no option but to comply with the detailed balance condition.''
  
   Is this a speculative sentence or is there an available proof for
   this? If the right choice is the latter, please give a reference and a
   precise statement of what the authors want to say (it seems quite a
   strong statement, given the equivalence between particle/hole models
   and Ising models.) If it is the former, give some hint of the
   physical/mathematical reason or delete the sentence. }
  

This strong statement not speculation: the proof is part of our work 
and one of our non-trivial
results.  We are not aware of a pre-existing proof, and because it is
quite technical we presented our proof in the SM.  In the main paper
now are more specific about the class of models which we refer to.

  {\it
   b) Given that detailed balance holds, a certain effective ``energy'' can
   be written for the model, which the authors state that are
   ``proportional to the number of particle-particle adjacencies in the
   system''.
     If I am getting this right, what the authors say in spin variables
   ($n_i=(1+\sigma_i)/2$ can be rewritten as the usual Ising model with
   nearest neighbor interactions   in the presence of an external field. }

As we said in the abstract, the Ising Hamiltonian does apply to the
system. However, there is no field, either in the Ising sense or
driving the particles.

{\it
Since the proposed rates verify detailed balance, I suspect that the
   authors are considering a variation of the Metropolis (or Glauber's)
   transition rates.}

This is a particle model, not a spin one, so away from the boundaries
the dynamics conserve particle number (i.e. they are not Glauber).
The Gillespie algorithm is similar to Metropolis, but more efficient
when there are a wide range of rates.  Also, there is no term
equivalent to an external field: an important feature of the model is
that there are no symmetry-breaking local driving forces, such as a
field might provide.

{\it   Is this the correct picture? If so, it seems a bit exaggerated to
   speak of a ``new'' model and it would be fairer to speak of a new
   ``interpretation'' of a previously existing model.}
  

In equilibrium, our dynamics must reproduce the Ising model because of
detailed balance. Indeed, we did verify the Ising-like equilibrium
behaviour, but only to test the coding.  This test could fairly be
described as a ``new ``interpretation'' of a previously existing
model'', so we didn't think that result interesting enough to include
in PRL.  To avoid confusion, we now mention it.


Perhaps the suspicion the we are simply solving the
Ising model with novel dynamics
is behind the comment: 
 ``I would not say that this work represents a substantial advance''.  
  In fact, we do much more than this.


  The novelty is to show that non-trivial flow behaviour does not
  require violating detailed balance.  Previous 1D models have been
  devised which induce non-equilibrium behaviour by breaking detailed
  balance. We demonstrate that this is not necessary.
 
  For that reason our model is not simply another non-equilibrium 1D
  flow model.  The local physics are equilibrium dynamics the only
  asymmetry driving the flow is at the boundaries, well away from
  where the transition occurs.
 
  We are aware that PRL discourages use of ``new'' and avoid the word.
 
{\it   
   b) Sticky particles in 1d have been thoroughly investigated. The
   literature is so large that it is difficult to give specific
   references, but the authors may have a look at:
  \begin{itemize}
   \item S. F. Shandarin and Y. B. Zeldovich, Reviews of Modern Physics 61, 185 (1989)
   \item L. Frachebourg, Phys. Rev. Lett. 82, 1502 (1999)
   \item L. Frachebourg, P. Martin, and J. Piasecki, Phys. A. 279, 69 (2000)
   \item E. Ben-Naim, S. Y. Chen, G. D. Doolen, and S. Redner, Phys. Rev.
   Lett. 83, 4069 (1999)
  \end{itemize}
  
   The last paper deals with a granular gas in 1d, which is shown to
   behave (for long enough times) in a manner completely analogous to a
   perfectly sticky gas (second and third papers). May the model proposed
   by the authors be connected to the granular gas in some (approximate)
   way?}

These and many similar papers provide examples of simple
driven-dissipative models which reduce to the Burgers equation in the
mean field limit.  Their dynamics is driven by local interactions
which do not obey detailed balance and their non-trivial behavior
comes from the interplay between dissipation and driving. 

We cite them, but emphasize that our model is of a completely
different type: locally it is an equilibrium (Ising) system with non
symmetry-breaking field, the driving comes only from the boundaries.

  {\it 
   2) The general presentation is definitely improvable. Some comments,
   not trying to be exhaustive, which show that the preparation has not
   been careful enough:
  
   a) $x_0$ in (5) has not been defined previously.
  
   b) $J_0$ in (5) is a constant to be determined or it is supposed to
   have a specific expression at this point? In (6), I wonder whether $J$
   should be $J_0$.
  
   c) Before (5), the authors say "and by integrating both sides of that
   equation with respect to x we find that (...)"; which equation are the
   authors referring to, the inline equation $J(x)=J_0$ or a different
   one?
  
   d) In page 3, $\rho_B$ and $rho_T$ are introduced for
   $\rho_M\pm\delta\rho$. They were called before $\rho_0$ and $\rho_L$
   in (6), if I have not lost the thread. Why the change of notation?
   Curiously, again $\rho_0$ and $\rho_L$ are defined in the following
   paragraph.
  
   e) The term ``block'' is repeatedly used but it is not defined, for
   example when below (6) the authors state ``We may consider applying
   small concentration gradients across a block (\dots) Doing so, we find
   that the effective diffusion coefficient of the block $D_{Eff}=\dots$
   (\dots)'' Is a block the whole system (the boundary conditions are at 0
   and L) or a part of the system?
  }

We have unified the notation and defined all quantities before use.  
Since it caused confusion, we decided to drop the term ``blocked state''.  


  {\it 3) After (7), can the authors comment something on the
    expression they have just derived?  } 

The main feature is that even for the weakly driven limit, the
equation has the same negative diffusion solutions and symmetry around
$\rho=2/3$ as equation 4, so we have referred back to equation 4.
  
{\it    4) In the first paragraph of the article, after the sentence ``This is
   unfortunate because it is precisely these nonlinearities which
   introduce interesting behaviors such as discontinuities at the
   oxide-metal interface or diffusion instability.'', suitable references
   must be provided.}
  
   References for this are added.  It was our work on TiO2 interfaces
   was what motivated the current model, although this historical
   detail may be of limited interest to the reader.
  
{\it   5) System sizes are very small, 32, 64, 124 sites. In Fig. 2, the
   authors say that sizes 32 and 64 give similar results. The authors
   should include some explicit comparison of the results for different
   sizes. Why the larger system in Fig. 4 (124 sites)?}
  
   - For practical purposes, the system sizes are not small and the
   work has required millions of CPU hours. Each point in Fig 1
   involved over 100 million Gillespie steps. Given that the number of
   Gillespie steps required for equilibration scales aggressively in
   the system size (to the third power in the number of lattice
   sites), we have to pay dearly for any increase in system size. We
   have repeated our 64-size calculations at size 32, 128 and 256, and
   they are in agreement.  These are not shown, because the data
   points are sparser and the larger calculations with larger lambda
   suffer from noise because because they take longer to converge.
   Thus, we are satisfied that size 64 is the best compromise in terms
   of getting good statistics with minimal edge-effect.

  
{\it    6) Below Fig. 2, it is reported that ``Firstly, we should note that we
   only have an MFT prediction for the flow rate as a function of
   $\lambda$, since $\rho(x)$ stops being unique when $\lambda$ drops
   below 1/4 and so the MFT lacks predictive power. For low-stickiness,
   when $\lambda 1/4$, the MFT is in good agreement with the
   simulations.''
  
   How is this conclusion drawn by the authors?}
  
   - One can see from the plots that the MFT is a good fit to the data
   when lambda 0.25. The result about the non-unique density follows directly from the cubic form of Eq.5, a full derivation is now added to the SM.
  

{\it 
   Minor comments:
  
   7) This is certainly a matter of taste, but in relation to the kinetic
   Monte Carlo integration I would cite the following papers
  
   - A. B. Bortz, M. H. Kalos, J. L. Lebowitz, J. Comput. Phys. 17, 10
   (1975).
  
   - A. Prados, J. J. Brey and B. Sánchez-Rey, J. Stat. Phys. 89, 709-734
   (1997).
  
   Instead of (or in addition to) Gillespie's. The first one introduces
   the n-fold way employed by KMCLIB, the second one shows that this kind
   of KMC algorithms solve the master equation numerically.
  }

   We agree, these are useful background to the algorithm, so we have included them.
  
{\it    8) I would definitely start a new paragraph with the sentence ``We have
   used this setup to explore three scenarios, discussed in the following
   sections.'' in the paragraph below (7), changing ``this setup'' for ``the
   setup above'', for example.}
  
   We have done this.
  
 


\section{}

\end{document}
